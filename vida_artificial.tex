\batchmode
\usepackage{authblk}
\usepackage{natbib}
\documentclass[conference]{IEEEtran}
\begin{document} 

\title{Vida artificial: Simulaci\'on de vida artificial y su impacto en el estudio de la biodiversidad }


% author names and affiliations

\author{\IEEEauthorblockN{David S\'anchez Alb\'an,   Natalia Mar\'in P\'erez}
\IEEEauthorblockA{Ingenier\'ia en Computaci\'on\\
Instituto Tecnol\'ogico de Costa Rica\\
San Jos\'e, Costa Rica}

}


% make the title area
\maketitle


\begin{abstract}
%\boldmath
La vida y su interacci\`on con el ser humano siempre ha sido de gran interes a la hora de realizar estudios en la materia. 
bla bla bla... 



\end{abstract}

\section{Introducci\'on}

La vida artificial o a-life se decdica a la creacion y al estudio de organismos y sistemas construidos por humanos. \cite{artificiallifelevy} 

Es por esto que se nos da la tarea de poder emular vida computacional, con el fin de entenderla. Ya que creando vida es una de las maneras de las que lograremos entenderla. \cite{artificiallifelevy}  

\section{Marco Te\'orico}

% En el doc que tengo en drive tengo bastanteeeee que poner aca 

\section{Simulaci\'ones y su impacto en el ecosistema}

% aqui comentamos sobre los estudios que hemos leido? 
% podriamos compararlos de alguna manera? 

\section{Posible contribuci\'on de Costa Rica en vida artificial}

% Hay contribuciones de costa rica en la vida artificial? 

\section{Impacto en Costa Rica}

% Hay papers de esto? 

\section{Trabajo relacionado}

% esto no deberia despues del marco teorico? 

\section{Conclusi\'on}

% ? hmm? emmm? Tenemos conclusiones, o vamos adecir que la vida artificial es un tema bonito. jaja 
\nocite{*}

\bibliographystyle{IEEEtran}
\bibliography{vida_artificial}

\bibitem{aritificallifelevy}
	Levy, Steven 
	\emph{Artificial Life}
	1992. 		


\end{document}
