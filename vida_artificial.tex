\documentclass[conference]{IEEEtran}

%\usepackage{natbib}
\usepackage{multirow}
\usepackage{mathtools}
\usepackage{fixltx2e}
\usepackage{amssymb}
\usepackage{graphics}
\usepackage{balance}

\title{Vida artificial: Simulaci\'on de vida artificial y su impacto en el estudio de la biodiversidad }

% author names and affiliations

\author{\IEEEauthorblockN{David S\'anchez Alb\'an,   Natalia Mar\'in P\'erez}
\IEEEauthorblockA{Ingenier\'ia en Ciecias de la Computaci\'on\\
Instituto Tecnol\'ogico de Costa Rica\\
San Jos\'e, Costa Rica}

}
\begin{document} 

% make the title area
\maketitle


\begin{abstract}
%\boldmath
La vida y su interacci\'on con el ser humano siempre ha sido de gran inter\'es a la hora de realizar estudios en la materia. 
bla bla bla... 



\end{abstract}

\section{Introducci\'on}

Vida artificial fue definida por Chris Langton como `el estudio de sistemas hechos por el hombre que exhibe comportamientos caracter\'isticos de sistemas vivos provenientes de la naturaleza' \cite{JDOY01}, es un \'area de estudio bastante reciente que puede guiar las investigaciones cient\'ificas en la manera de extender la vida y crear nuevas formas de ella, incluyendo medicinas, internet, hardware que pude evolucionar y la proliferaci\'on de robots. \cite{PROB01}\\
Es por esto que se nos da la tarea de poder simular vida a trav\'es de la computaci\'on para crear vida y de esta manera entenderla. \cite{artificiallifeLevy}  \\
Para crear vida artificial que sea robusta por computadora, es necesario que esta pueda sobrevivir las fluctuaciones del ambiente y evolucionar tan libremente como su vida biol\'ogica. El software debe poder adaptarse con algoritmos de aprendizaje que permitan a los programas de computadora ganar experiencia, as\'i como programas que sean capaces de escribir otros programas de computadoras con un comportamiento de "b\'usqueda de metas" que permita a los programas funcionar en ambientes espec\'ificos. El software de computadora debe poder innovar y agregar en s\'i mismo la respuesta a sus "necesidades". El solucionar estos problemas es una de las metas principales en el estudio de vida artificial \cite{JDOY01}.\\
En el an\'alisis acerca de herramientas de vida artificial escrito por Steven Levy se explica el de un sistema desarrollado por el bi\'ologo Thomas Ray el cual plantea una herramienta llamada "Tierra". Una vez que el sistema fue finalizado, este pod\'ia cambiar su criterio por el cual se constitu\'ia un organismo apto, y cuando este se llenaba de organismos el ambiente evolutivo cambiaba tambi\'en; las criaturas digitales fueron forzadas a buscar respuestas novedosas cuando las circunstancias eran alteradas. Esto se lograba gracias a sistemas de "reconocimiento", ya que de los contrario habr\'ian mutaciones que no se llevar\'ian a cabo. El sistema est\'a buscando constantemente en hacer el c\'odigo eficiente, pero por otra parte la evoluci\'on se da al "explotar" entre s\'i, los organismos agregan el concepto del m\'as apto, una nueva adaptaci\'on que se da al transmitir los genes que puede contener un mecanismo espec\'ificio que no necesariamente est\'a presente en el ancestro. Tierra refleja comunidades ecol\'ogicas al simular un depredador el cual va a suprimir al competidor y lo excluye como uno de los competidores d\'ebiles impactando as\'i en la diversidad. \cite{STEV01} \\

Una vez que se entiende una forma de vida y su comportamiento es posible ayudar en c\'omo estos podr\'ian impactar el ecosistema en que vivimos y entender mejor de que manera los seres vivos impactan en el medio ambiente.



\section{Marco Te\'orico}

% En el doc que tengo en drive tengo bastanteeeee que poner aca 


A la hora de poder desarrollar, estudiar o crear vida artificial, se debe de tener un entendimiento de la vida en s\'i, donde las m\'ultiples \'areas de la academia han intentado definir, qu\'e es? Los fil\'osofos utilizan t\'erminos para discernir entre lo vivo y lo no-vivo, y son estas cualidades lo que hace a algo pertenecer al \'area de los entes vivos. \cite{lifeStanfordPhi} En el area de la biolog\'ia se utiliza la reproducci\'on y la supervivencia \cite{artificiallifeLevy} como capacidades necesarias con el fin de definir algo como vivo. 

La vida es considerada org\'anica, ya que esta surge naturalmente y es un concepto irremplazable del mundo natural, la cual es un \'area de estudio para los bi\'ologos y dem\'as \'areas de la ciencia y tecnol\'ogia. El concepto de vida ha sido estudiado por cientos de a\~nos, pero siempre existen conflictos a la hora poder definir una definici\'on concreta, por ejemplo: Arist\'oteles defini\'o la vida como la propiedad de un objeto de ser animado, Descartes como un mecanismo, el punto de vista de Kant como una organizaci\'on. \cite{lifeStanfordPhi} \\
Tambi\'en es importante entender lo que es la vida natural, la cual tiene las siguientes caracter\'isticas \cite{XUY01}:
\begin{itemize}
\item Crecimiento natural, evoluci\'on y no hecho por el hombre.
\item Reproducci\'on sexual, por ejemplo, humanos, otros animales y plantas
\item Basado en prote\'inas, sustancias org\'anicas.
\item Inteligencia y emociones, tales como el humano u otros animales
\end{itemize}

Pero qu\'e pasa cuando la biolog\'ia y las ciencias de la computaci\'on se mezclan, con esto surge la pregunta, ser\'a el poder computacional actual capaz de emular las cualidades necesarias para crear vida artificial? A esto se ha llamado vida `in-silico' \cite{artificiallifeLevy, lifeStanfordPhi} esto por el uso de los chips semiconductores necesarios para el uso de software. El uso de la vida in-silico se debe primariamente a la gran capacidad de procesamiento que poseen las computadoras para evaluar modelos complejos sobre vida artificial, a parte de poder ayudarnos a mejorar el concepto de vida artificial que tenemos. 

Parte de la teor\'ia que podria mostrar el flujo de la vida seria el uso de teor\'ia de automatas, para poder crear una visualizacion la cual satisfaga todas las opciones que sean necesarias para mostrar vida bajo una definicion especifica. Para esto se podria usar una Maquina de Estados Finitos (FSM) \cite{MARG01} la cual nos ayude a demostrar una serie de estados en la cual un organismo puede estar, pero, esto generar\'ia un FSM demasiado grande, el cual seria inmanejable para un ser humano, pero una computadora podria re-crear un ente sencillo, d\'igase de una bacteria o un insecto.  
% Expander y revisar ortografia, no tengo acceso a tildes! D: 
%tildes agregadas

En el a\~no 1982, el cient\'ifico Stephem Wolfram explor\'o y categoriz\'o los tipos de complejidad que mostraban los autómatas celulares unidimensionales, y se dieron cuenta que estos podr\'ian ser aplicados a fen\'omenos naturales como las conchas marinas y la naturaleza del crecimiento de las plantas. Tambi\'en, Norman Packard utiliz\'o los aut\'omatas celulares para simular el crecimiento de copos de nieve. \cite{VAD01} \\
Existen dos posiciones en vida artificial\cite{VAD01} :
\begin{itemize}
\item La posici\'on fuerte/dura que indica que "la vida es un proceso que se puede conseguir fuera de cualquier medio particular". (John Von Neumann). Como se indicaba anteriormente en el sistema Tierra donde la vida era sintetizada seg\'un Thomas Ray.
\item La posici\'on d\'ebil la cual niega la posibilidad de generar un "proceso de vida" fuera de una soluci\'on qu\'imica basada en el carbono, en cambio se opta por imitar procesos de vida para entender aspectos de fen\'omenos sencillos.
\end{itemize}


Un claro ejemplo del uso de vida artificial es el uso de modelos complejos con el fin de evaluar los resultados y obtener una simulaci\'on para satisfacer las pruebas necesarias, estos modelos pueden ayudarnos a explicar un comportamiento espec\'ifico. Tomemos el caso de las abejas arboricuas Apis mellifera, las cuales recolectan polen con el fin de transformarlo en miel y mantener la supervivencia de su colonia, estas poseen un comportamiento interesante a la hora de escojer las flores adecuadas, debido a que solamente estas proporcionan el polem adecuado para producir su preciada miel. \cite{ZOE01} El uso de vida artificial es imprecidible para poder crear un ambiente digital en el cual las abejas virtuales o agentes puedan interactuar con su medio, asimismo se pueden evaluar ambientes mas complejos y reducir el trabajo de campo. 




1124

hormigas 151 - paper de naty

Este paper es un estudio alrededor de las hormigas  Leptothorax tuberointerruptus,\cite{LP01}, las cuales formaron parte esencial a la hora de realizar la investigacion para porponer un modelo consistente y continuo el cual permite controlar un algoritmo el cual cree arquitecturas nuevas con el fin de demostrar el aprendizaje. % @ intro

mosuqitos 657

flock 1114

192

488




\section{Simulaci\'ones y su impacto en el ecosistema}

% aqui comentamos sobre los estudios que hemos leido? 
% podriamos compararlos de alguna manera? 

126 - evolved ecosystems

Aqui podemos comprar los estudios relacionados con el estudio de organismos: hormigas, abejas, pajaros (flock) 

\section{Posible contribuci\'on de Costa Rica en vida artificial}

% Hay contribuciones de costa rica en la vida artificial? 

Naty? 

\section{Trabajo relacionado}

Open worm es un estudio y/o proyecto el cual esta intentado re-crear una lombriz utilizando la vida artificial simulado por computadores.  Al simular las mil c\'elulas de la lombriz Caenorhabditis elegans (C. elegans) se lograr\'a entender comportamientos simples y complejos lo cual satisface lo suficiente para poder realizar un estudio compresivo al respecto, esta posee comportamientos clave como: alimentaci\'on, reproducci\'on, evitar depredadores, entre otras caracter\'isticas. \cite{openworm} Todas estas caracteristicas son vitales para poder entender el concepto filos\'ofico de qu\'e es la vida y cu\'ales son sus componentes b\'asicos para el estudio al respecto. \\
Aunque ha sido estudiado profundamente, a\'un hace falta tener un mayor entendimiento de los principios de su biolog\'ia.
Una vez que se logre la meta se espera que esto ayudar\'a a la creaci\'on de otras criaturas virtuales que sean igual de precisas.
Aparte de ser una herramienta sumamente importante, Open worm... % MUERO, http://docs.openworm.org/en/0.9/Community/github/, seguir ahi. 

\section{Conclusi\'on}

Existen estudios... es interesante porque... los apredido fue...

\nocite{*}

\bibliographystyle{IEEEtranS} % ordena por nombre de autor
\bibliography{vida_artificial}
\end{document}
