\batchmode
\usepackage{authblk}
\usepackage{natbib}
\documentclass[conference]{IEEEtran}
\begin{document} 

\title{Vida artificial: Simulaci\'on de vida artificial y su impacto en el estudio de la biodiversidad }


% author names and affiliations

\author{\IEEEauthorblockN{David S\'anchez Alb\'an,   Natalia Mar\'in P\'erez}
\IEEEauthorblockA{Ingenier\'ia en Ciecias de la Computaci\'on\\
Instituto Tecnol\'ogico de Costa Rica\\
San Jos\'e, Costa Rica}

}


% make the title area
\maketitle


\begin{abstract}
%\boldmath
La vida y su interacci\`on con el ser humano siempre ha sido de gran interes a la hora de realizar estudios en la materia. 
bla bla bla... 



\end{abstract}

\section{Introducci\'on}

La vida artificial o a-life se dedica a la creaci\`'on y al estudio de organismos y sistemas construidos por humanos. \cite{artificiallifeLevy} 

Es por esto que se nos da la tarea de poder emular vida computacional, con el fin de entenderla. Ya que creando vida es una de las maneras de las que lograremos entenderla. \cite{artificiallifeLevy}  

\section{Marco Te\'orico}

% En el doc que tengo en drive tengo bastanteeeee que poner aca 


A la hora de poder desarrollar, estudiar o crear vida artificial, se debe de tener un entendimiento por de que es la vida en si, donde las m\`ultiples \`areas de la academia han intentado definir, que es? Los fil\`osofos utilizan t\`erminos para discernir entre lo vivo y lo no-vivo, y son estas cualidades lo que hace a algo pertenecer al \`area de los entes vivos. \cite{lifeStanfordPhi} En el area de la biolog\`ia se utiliza la reproducci\`on y la supervivencia \cite{artificiallifeLevy} como capacidades necesarias con el fin de definir algo como vivo. 

La vida es considerada org\`anica, ya que esta surge naturalmente y es un concepto irremplazable del mundo natural, la cual es un \’area de estudio para los bi\’ologos y dem\’as \`areas de la ciencia y tecnol\`ogia. El concepto de vida ha sido estudiado por cientos de a\~nos, pero siempre existen conflictos a la hora poder definir una deficinici\`on concreta, por ejemplo: Arist\’oteles defini\’o la vida como la propiedad de un objeto de ser animado, Descartes como un mecanismo, el punto de vista de Kant como una organizaci\’on. \cite{lifeStanfordPhi} 

Pero que pasa cuando la bi\’ologia y las ciencias de la computaci\’on se mezclan, con esto surge la pregunta, ser\’a el poder computacional actual capaz de emular las cualidades necesarias para crear vida artificial? A esto se ha llamado vida `in-silico' \cite{artificiallifeLevy, lifeStanfordPhi} esto por el uso de los chips semiconductores necesarios para el uso de software. El uso de la vida in-silico se debe primariamente a la gran capacidad de procesamiento que poseen las computadoras para evaluar modelos complejos sobre vida artificial, a parte de poder ayudarnos a mejorar el concepto de vida artificial que tenemos. 

FSM

Un claro ejemplo del uso de vida artificial es el uso de modelos complejos con el fin de evaluar los resultados y obtener una simulaci\’on para satisfacer las pruebas necesarias, estos modelos pueden ayudarnos a explicar un comportamiento espec\’ifico. Tomemos el caso de las abejas Apis mellifera, las cuales recolectan polen con el fin de transformarlo en miel y mantener la supervivencia de su colonia, estas poseen un comportamiento interesante a la hora de escoger las flores adecuadas, debido a que solamente estas proporcionan el polen adecuado para producir su preciada miel. \cite{ZOE01} El uso de vida artificial es imprescindible para poder crear un ambiente digital en el cual las abejas virtuales o agentes puedan interactuar con su medio, asimismo se pueden evaluar ambientes más complejos y reducir el trabajo de campo. 



1124

hormigas 151 - paper de naty

mosuqitos 657

flock 1114

192

488




\section{Simulaci\'ones y su impacto en el ecosistema}

% aqui comentamos sobre los estudios que hemos leido? 
% podriamos compararlos de alguna manera? 

126 - evolved ecosystems

\section{Posible contribuci\'on de Costa Rica en vida artificial}

% Hay contribuciones de costa rica en la vida artificial? 

\section{Impacto en Costa Rica}

% Hay papers de esto? 

\section{Trabajo relacionado}a

% esto no deberia despues del marco teorico? 

\section{Conclusi\'on}

% ? hmm? emmm? Tenemos conclusiones, o vamos adecir que la vida artificial es un tema bonito. jaja 
\nocite{*}

\bibliographystyle{IEEEtran}
\bibliography{vida_artificial}
\end{document}
